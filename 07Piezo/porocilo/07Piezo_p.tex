\documentclass[12pt]{report}

\usepackage[a4paper]{geometry}
%\geometry{left=2.5cm,right=2.5cm,top=2.5cm,bottom=2.5cm, a4paper}
\usepackage[utf8]{inputenc}
\usepackage{amsmath}
\usepackage{amsthm}
\usepackage{amssymb}
\usepackage{ulem}
\usepackage{graphicx}
\usepackage{caption}
\graphicspath{}
\usepackage[document]{ragged2e}
\usepackage{setspace}
\usepackage{tabularx}
\usepackage[slovene]{babel}
\usepackage{textcomp, gensymb}
\usepackage{siunitx}
\usepackage{pdfrender,xcolor}
\usepackage{hyperref}
\usepackage{xurl}
\usepackage{float}
\usepackage{titlesec}

\newfloat{slika}{htbp}{loc}
\floatname{slika}{Slika}

\newfloat{tabela}{htbp}{loc}
\floatname{tabela}{Tabela}

\title{
  \includegraphics[width=0.4\textwidth]{fmf_logo}\\
  {\small Oddelek za fiziko} \\
  {Piezoelektričnost}\\
  {\small Poročilo pri fizikalnem praktikumu III}\\

}
\date{}
\author{ Kristofer Č. Povšič \\[5 cm]
 \small  Asistentka: Jelena Vesić
}


\titleformat{\chapter}[hang]{\Huge\bfseries}{\thechapter{. }}{0pt}{\Huge\bfseries}

\setlength\parindent{0pt}

\begin{document}

\setcounter{page}{2}

\maketitle

\chapter*{Uvod}

Kristali v feroelektričnem stanju so tudi piezoelektrični: mehanska obremenitev spremeni električno polarizacijo in obratno, zunanje električno polje, v katerem je kristal, povzroči deformacijo kristala. Vzrok zaa to je sklopitev med mehansko in električno energijo kristala. 

Piezoelektrični efekt dobimo pri kristalih, ki nimajo centra simetrije. 

Piezoelektrično se odzovejo na deformacijo s polarizacijo snovi. Lokalno deformacijo v točki s silo $\text{d}\vec{F}_i$, $i = 1, 2, 3$, podano z napetostnim tenzorjem $T_{i,j}$, $i,j = 1,2,3$ defofinirano kot
\begin{equation}
  T_{i,j} = \frac{1}{2}\left(\frac{\text{d}\vec{F}_i}{\text{d}S_j} + \frac{\text{d}\vec{F}_j}{\text{d}S_i}\right)
\end{equation}

kjer so $\text{d}S_i$, $i=1,2,3$ tri med seboj pravokotno orientirane površine delovanja sile $\text{d}\vec{F}$. Elemente tenzorja $d_{i,j,k}$ imenujemo "piezoelektrični moduli". Praktično jih merimo v izbranih smereh kristala. Električni naboj merimo na izbrani ploskvi, mehanska obremenitev pa je lahko tlak na ploskev, upogib ali torzija. Kot piezoelektrik največkrat uporabimo kvarc v obliki ploščic ali piezoleketrično keramiko, v obliki ploščic, ki so polarizirano pravokotno na ploskev. Uporabljamo jih za merjenje tlaka in sil (mikrofoni/generatorji ultrazvoka). 

Pomembni pa so tudi za delovanje vrstičnega tunelskega mikroskopa. 

Pri vaji bomo izmerili odziv piezoelektričnih ploščic in piezoelektrične keramike. V tem primeru ima tenzor $d_{i,j,k}$ tri neodvisne elemente. To so $d_{121}$, $d_{311}$, $d_{333}$, če je os z izbrana vzporedno z začetno polarizacijo keramike. S silo $\vec{F}$, pravokotno na $S$ ustvarimo tlačno napetost $T=T_{33} = \frac{F}{S}$ in povzročimo nastanek polarizacije $P_3 = \text{d}T$. V snovi vzdolž z-osi velja med $P_3$ in gostoto električnega polarizacije
\begin{equation}
  D = \varepsilon \varepsilon_0 E + \text{d}T
\end{equation}

kjer je $\varepsilon$ električna konstanta pri konstantni napetosti, temperaturi in $d=d_{333}$. Naboj na eni ploskvi kondenzatorja oz. ploščice keramike je $q = DS$ in z upoštevanjem povezave med električno jakostjo in napetostjo $E= \frac{U}{b}$ dobimo, da je naboj

\begin{equation}
  q = \frac{\varepsilon \varepsilon_0 S}{b} U + \text{d}F
\end{equation}

Opazimo, da je 1. člen le drug način, da je napišemo na ploščatem kondenzatorju s ploščino S in debelino b s kapaciteto: 
\begin{equation}
  C = \frac{\varepsilon \varepsilon_0 S}{b}
\end{equation}

Torej 

\begin{equation} 
  q = CU + text{d}F
\end{equation}

Kondenzator praznimo preko uporanika $R = 5G\omega (1 \pm 0.02)$  s tokom $I = -\dot{q}$. Padec napetosti

\begin{equation}
  U = RI = -R\dot{q}
\end{equation}

Spremljamo na osciloskopu preko povratno vezanega ojačevalnika (voltage follower). Dobimo diferencialno enačbo za časovni potek napetosti

\begin{equation}
  \dot{U} = -\frac{1}{\tau} U - \frac{d}{C}\dot{F}
\end{equation}

kjer je $\tau = RC$. Sedaj obravnavamo oba predvidena scenarija spremembe napetosti na kristalu, za katere je 
\begin{equation}
  F_s(t) = F_0 \theta(st)\\
  \theta = 
  \begin{cases}
    1, & \ t \geq 0 \\
    0, & \text{sicer}
  \end{cases}
\end{equation}

kjer je $F_0$ teža uteži, $s$ pa pozitiven, če prihaja do obremenjevanja in negativen

\chapter*{Naloga}


\begingroup
\let\clearpage\relax

\chapter*{Potrebščine}
\begin{itemize}
\item 
\end{itemize}

\chapter*{Navodilo}

\endgroup


\chapter*{Obdelava podatkov}



\chapter*{Izračunane vrednosti}


\chapter*{Komentar}



\end{document}