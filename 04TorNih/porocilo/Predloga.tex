\documentclass[12pt]{report}

\usepackage[a4paper]{geometry}
%\geometry{left=2.5cm,right=2.5cm,top=2.5cm,bottom=2.5cm, a4paper}
\usepackage[utf8]{inputenc}
\usepackage{amsmath}
\usepackage{amsthm}
\usepackage{amssymb}
\usepackage{ulem}
\usepackage{graphicx}
\usepackage{caption}
\graphicspath{}
\usepackage[document]{ragged2e}
\usepackage{setspace}
\usepackage{tabularx}
\usepackage[slovene]{babel}
\usepackage{textcomp, gensymb}
\usepackage{siunitx}
\usepackage{pdfrender,xcolor}
\usepackage{hyperref}
\usepackage{xurl}
\usepackage{float}
\usepackage{titlesec}

\newfloat{slika}{htbp}{loc}
\floatname{slika}{Slika}

\newfloat{tabela}{htbp}{loc}
\floatname{tabela}{Tabela}

\title{
  \includegraphics[width=0.4\textwidth]{fmf_logo}\\
  {\small Oddelek za fiziko} \\
  {Torzijsko nihalo}\\
  {\small Poročilo pri fizikalnem praktikumu III}\\

}
\date{}
\author{ avtor: Kristofer Č. Povšič \\[5 cm]
 \small  Asistentka: Jelena Vesić
}


\titleformat{\chapter}[hang]{\Huge\bfseries}{\thechapter{. }}{0pt}{\Huge\bfseries}

\setlength\parindent{0pt}

\begin{document}

\setcounter{page}{2}

\maketitle

\chapter*{Uvod}

O strižni napetosti $\frac{F}{S}$ govorimo takrat, ko leži sila v ravnini ploskve, v kateri prijemlje, v nasprotju s tlačno silo, ki je pravokotna na ploskev. Za strižno napetost velja enačba:

\begin{equation}
  \frac{F}{S} = G \alpha
\end{equation}

Torzijski koeficient žice je sorazmernostni faktor med navorom in kotom $\varphi$, zasukom prostega konca žice: 

\begin{equation}
  M = D \varphi
\end{equation}

Torzijski koeficient je za zasuke, ki niso preveliki, povezan s strižno napetostjo G preko 

\begin{equation} \label{eq:2}
  D = \frac{\pi \rho^4G}{2l}
\end{equation}

pri čemer je $\rho$ polmer, $l$ pa dolžina žice. Če na spodnji konec viseče žice obesimo telo in ga zasukom, lahko opazujemo torzijsko nihanje. Za majhne zasuke je to nihanje harmonično in velja: 

\begin{equation} \label{eq:1}
  t_0= 2\pi \sqrt{\frac{J}{D}}
\end{equation}

\chapter*{Naloga}

\begin{enumerate}
  \item Določi torzijski koeficient D žice
  \item Izračunaj strižni modul G jekla, iz katerega je žica
  \item Določi vztrajnostni moment in vztrajnostni radij danega telesa (kvadra z valjasto votlino) iz meritve nihajnega časa torzijskega nihala in primerjaj rezultate z izračunanim vztrajnostnim momentom. 
  \item Določi vztrajnostni moment zobnika. 
\end{enumerate}


\begingroup
\let\clearpage\relax

\chapter*{Potrebščine}
\begin{itemize}
\item stojalo, jeklena žica, plošča z ročajem
\item uteži: votel kovinski valj, kvader z valjasto votlino
\item tehtnica, štoparica, kljunasto merilo, mikrometer
\end{itemize}

\chapter*{Navodilo}
Izmeri dolžino in debeline žice nihaja. Nanjo obesi ploščo z ročajem. Izmeri nihajni čas nihala pri prazni plošči, votlim valjem in še s kvadrom z valjasto odprtino ter zobnik. Pri vsaki meritvi izmeri čas 10 nihajev s štoparico. Kotne amplitude pri nihanju naj bodo majhne. 
\endgroup


\chapter*{Obdelava podatkov}

Izmerjeni so bili sledeči podatki: 

Debelina žice $d = (0.63 \pm 0.05)mm$ in dolžina žice $l = (29 \pm 0.5)cm$. 

Izmerim in izračunamo naslednje povprečne čase za prazno ploščo, ploščo z valjem, ploščo s kvadrom in ploščo z zobnikom: 

\begin{tabela}[H]
  \centering
  \[
  \begin{array}{|c|c|c|c|c|}\hline
    \text{meritev} & t_p[s] & t_v[s] & t_k[s] & t_z[s] \\ \hline
    1 & 1.874 & 5.263 & 3.666 & 2.943 \\ \hline
    2 & 1.888 & 5.317 & 3.624 & 2.986 \\ \hline
    3 & 1.897 & 5.300 & 3.636 & 2.895 \\ \hline
    \\ \hline
    \overline{t}[s] & 1.886 & 5.293 & 3.642 & 2.941 \\\hline
    \sigma_t[s] & 0.009 & 0.023 & 0.018 & 0.037 \\ \hline
  \end{array}
  \]
\end{tabela}

Z izmerjenimi podatki valja: masa valja $m_v = (2500 \pm 1)g$, manjši radij $r = (15 \pm 87.2)mm$, veliki radij $R = (87.2 \pm 0.1)mm$ in višino $h = (50 \pm 0.1)mm$. 

Iz teh podatkov izračunam vztrajnostni moment valja $J = (2.446 \pm 0.007)gm^2$. Za nihalo z valjem po enačbi \ref{eq:1}: 
\begin{equation}
  D = (J_p + J_v)\left(\frac{2\pi}{t_v}\right)^2
\end{equation}
za $J_p$ pa tudi velja $J_p = D (\frac{t_p}{2\pi})^2$. 

Tako izračunamo torzijski koeficient žice $D = (3.94 \pm 0.04)mNm$. Iz enačbe \ref{eq:2} izračunamo strižni koeficient $G = (7 \pm 1)kN/mm^2$. 

Analogno velja tudi za poljubno telo, ki ga položimo na ploščo tako, da os vrtenja prebada njegovo težišče. Za kvader z luknjo tako dobimo rezultat $J_k = (0.97 \pm 0.02)gm^2$ in za zobnik $J_z = 0.53 \pm 0.01gm^2$. 

Za kvader z luknjo sem izmeril sledeče rezultate: masa kvadra $m_k = (1193\pm1)g$, stranica $a = (60 \pm 0.05)mm$ in premer valja $d = (40 \pm 0.05)mm$. 

Vztrajnostni moment za kvader se tako zapiše: 

\begin{equation}
  J_k = \sigma\left(\frac{S_k}{12}(2a^2) - \frac{S_v}{2}\left(\frac{d}{2}\right)^2\right)
\end{equation}

Iz enačbe poračunamo $J_k = (0.98 \pm 0.002)gm^2$. 

\end{document}